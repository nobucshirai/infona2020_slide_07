% [Overleaf] https://www.overleaf.com/read/cdmpqcngbknx
% [YouTube] https://youtu.be/XUWfTUkyRZQ
% [GitHub] https://github.com/nobucshirai/infona2020_slide_07
\documentclass[dvipdfmx,aspectratio=169,20pt]{beamer}
\usepackage{bxdpx-beamer}

% Beamer theme
\usetheme{Boadilla}

%%%%% JAPANESE FONT SETTINGS %%%%%
\usepackage[utf8]{inputenc}
\usepackage{pxjahyper}
\renewcommand{\kanjifamilydefault}{\gtdefault} % for Gothic Japanese fonts
\newcommand{\myfontsetting}[3]{{\fontsize{#1}{#2}\selectfont #3}}
\usepackage[deluxe,uplatex]{otf} % needed to use super bold Japanese fonts
\usepackage[unicode,noto-otc]{pxchfon} % needed to use super bold Japanese fonts
%%%%%%%%%%%%%%%%%%%%%%%%%%%%%%%%%%

%%%%% SETTINGS FOR MATH SYMBOLS %%%%%
\usepackage{amsmath,amssymb}
\usepackage{bm}
%\usepackage{graphicx}
\usepackage{latexsym}
\usefonttheme{professionalfonts} % use Serif fonts for equations
%%%%%%%%%%%%%%%%%%%%%%%%%%%%%%%%%%%%%

\usepackage{fancybox,ascmac}
\usepackage{url}
\usepackage[many]{tcolorbox}

%%%%% ALGORITHM SETTING %%%%%
\usepackage{algorithm}
\usepackage[noend]{algorithmic}
\algsetup{linenosize=\color{fg!50}\fontsize{8pt}{8pt}\selectfont}
\renewcommand\algorithmicdo{\bfseries :}
\renewcommand\algorithmicthen{\bfseries :}
\renewcommand\algorithmicrequire{\textbf{Input:}}
\renewcommand\algorithmicensure{\textbf{Output:}}
\renewcommand{\algorithmicprint}{\textbf{break}}
%%%%%%%%%%%%%%%%%%%%%%%%%%%%%
\definecolor{myblue1}{RGB}{45,130,200}
\definecolor{myblue2}{RGB}{26,89,142}
\setbeamertemplate{navigation symbols}{}
\setbeamercolor*{structure}{fg=myblue1,bg=blue}
\setbeamercolor{block title}{fg=myblue1!50!black}
\setbeamercolor*{block title example}{fg=white,bg=myblue2}
\setbeamercolor*{palette primary}{use=structure,fg=white,bg=structure.fg}
\setbeamercolor*{palette secondary}{use=structure,fg=white,bg=structure.fg!75!black}
\setbeamercolor*{palette tertiary}{use=structure,fg=white,bg=structure.fg!50!black}
\setbeamercolor*{palette quaternary}{fg=black,bg=myblue1}

\setbeamerfont{alerted text}{series=\bfseries}
\setbeamerfont{section in toc}{series=\mdseries}
\setbeamerfont{frametitle}{size=\Large,series=\bfseries}
\setbeamerfont{title}{size=\LARGE,series=\bfseries}
\setbeamerfont{date}{size=\small}

\setbeamertemplate{block title}[shadow=false]
\setbeamertemplate{blocks}[rounded][shadow=false]

%%%%% BEAMER FOOTLINE SETTINGS %%%%%%
\setbeamertemplate{footline}[frame number]{}
\setbeamerfont{footline}{size=\bf\footnotesize\small}
%%%%%%%%%%%%%%%%%%%%%%%%%%%%%%%%%%%%%

%%%%% BEAMER ITEM SETTINGS %%%%%
\setbeamertemplate{itemize item}[circle]
\setbeamertemplate{itemize subitem}[triangle]
\setbeamertemplate{enumerate item}[circle]
%%%%%%%%%%%%%%%%%%%%%%%%%%%%%%%%

\begin{document}

\graphicspath{{figs/}}

%%%%%%%%%%%%%%%%%%%%%%%%%%%%%%%%
\begin{frame}
%%%%% START_TAG B %%%%%
\frametitle{[問題] V\hspace{-.1em}I-B}
%\noindent{\bf V\hspace{-.1em}I-B.}

\myfontsetting{12pt}{12pt}{
$f(x)=e^x$, $x\in [-1,1]$ とした時、点 $(x_0, f(x_0)), (x_1, f(x_1)), \dots, (x_n, f(x_n))$ に対してニュートン補間を行い補間多項式 $p_n(x)$ を差分商を用いて表す場合を考える。
$x_0 = -1$, $x_n = 1$ とし $x_1, \dots, x_{n-1}$ を区間 $[-1,1]$ を $n$ 等分するように取った時、 $n=2,4,8,16,32$ の場合の $p_n(x)$ について相対誤差 $E(x)$ を区間 $[-1,1]$ を500等分する点 $(x=-1.0, -0.996, -0.992, \dots, -0.004, 0.0, 0.004, \dots, 0.992, 0.996, 1.0)$ で求め、$E(x)$ が最大となる $x$ とその時の $E(x)$ の値をそれぞれ有効数字10進3桁で4桁目を四捨五入して答えよ。
}\\
\myfontsetting{10pt}{10pt}{
作成したプログラムも提出すること。プログラミング言語は問わない。
}
%%%%% END_TAG B %%%%%
\end{frame}
%%%%%%%%%%%%%%%%%%%%%%%%%%%%%%%%
\begin{frame}
\frametitle{[略解] V\hspace{-.1em}I-B}
$n=2$ の時 $(x,E(x))=(-0.656, 0.108)$

$n=4$ の時 $(x,E(x))=(-0.840,1.96\times 10^{-3})$

$n=8$ の時 $(x,E(x))=(-0.928, 1.22\times 10^{-7})$

$n=16$ の時 $(x,E(x))=(-0.968, 6.72\times 10^{-14})$

$n=32$ の時 $(x,E(x))=(-0.988, 2.55\times 10^{-9})$

\end{frame}
%%%%%%%%%%%%%%%%%%%%%%%%%%%%%%%%
%タイトルページ

\title{関数近似と補間 (2)}
\titlegraphic{\vspace{-7mm}\flushright\includegraphics[width=1.8cm,height=1.8cm]{hattari_kun_good_org.eps}}

\setbeamertemplate{title page}{%
    \begin{flushright}
        \usebeamercolor[fg]{titlegraphic}\inserttitlegraphic
    \end{flushright}
    \vspace{-0.6cm}
    \hspace{1.5cm}{\selectfont\usebeamerfont{subtitle} \usebeamercolor[fg]{subtitle} [\href{https://youtu.be/XUWfTUkyRZQ}{数値解析 第7回}] \par}
    \vspace{0.5cm}
    %\vspace{2.5em}
    {\centering\usebeamerfont{title} \usebeamercolor[fg]{title} \inserttitle \par}
    \vspace{0.5cm}
    \begin{center}
        小区間に分けて早く滑らかにつなぐ
    \end{center}
}

\date[\todey]{}

\frame{\titlepage}

%%%%%%%%%%%%%%%%%%%%%%%%%%%%%%%%
\begin{frame}
\frametitle{{\Large 区分的多項式補間 \myfontsetting{12pt}{12pt}{(Piecewise polynomial interpolation)}}}

\begin{itemize}
    \setlength{\itemsep}{0.5cm}
    \item \myfontsetting{20pt}{20pt}{与えられたデータ点集合を分割した小区間毎に多項式補間する方法}
    \begin{itemize}
        \item \myfontsetting{15pt}{15pt}{小区画毎に通常の多項式補間をすれば原理的には区分的多項式補間になるが、小区間の境界で滑らに接続できない}
    \end{itemize}
    \item \myfontsetting{20pt}{20pt}{小区間の間を{\bf 滑らかに}つなげる区分的多項式補間}
    \begin{itemize}
        \setlength{\itemsep}{0.15cm}
        \item \myfontsetting{20pt}{20pt}{\bf スプライン補間} 
    \end{itemize}
\end{itemize}
\end{frame}
%%%%%%%%%%%%%%%%%%%%%%%%%%%%%%%%
\begin{frame}
\frametitle{{\large 3次スプライン補間 \myfontsetting{12pt}{12pt}{ (Cubic spline interpolation)}}}
\begin{itemize}
    \setlength{\itemsep}{0.5cm}
    \item \myfontsetting{15pt}{15pt}{隣り合うデータ点間を3次多項式で{\bf 滑らかに}つなげる方法}
    \begin{itemize}
        \item [\myfontsetting{12pt}{12pt}{\bf 問題設定}] \myfontsetting{12pt}{12pt}{ $n+1$個 のデータ点 $(x_k, f(x_k))$ \myfontsetting{10pt}{10pt}{ $(0\le k\le n)$} を $n$ 本の3次多項式 \myfontsetting{12pt}{12pt}{$p_i(x) = f(x_i) + a_i (x-x_i) + b_i (x-x_i)^2 + c_i (x-x_i)^3$}, \myfontsetting{10pt}{10pt}{$(0 \le i \le n-1)$} で補間する問題 ($a_i, b_i, c_i$ \myfontsetting{10pt}{10pt}{$(0 \le i \le n-1)$} の $3n$ 変数の値を求める)
        }
        \item [\myfontsetting{12pt}{12pt}{\bf 条件1}] \myfontsetting{12pt}{12pt}{$p_i(x_{i+1}) = f(x_{i+1})$, \myfontsetting{10pt}{10pt}{$(0\le i \le n-1)$}}
        \item [\myfontsetting{12pt}{12pt}{\bf 条件2}] \myfontsetting{12pt}{12pt}{$p_i^\prime(x_{i+1}) = p_{i+1}^\prime(x_{i+1})$ , $p_i^{\prime\prime}(x_{i+1}) = p_{i+1}^{\prime\prime}(x_{i+1})$} , \myfontsetting{10pt}{10pt}{$(0\le i \le n-2)$}
        \item [\myfontsetting{12pt}{12pt}{\bf 端条件}] \myfontsetting{12pt}{12pt}{$p_0^{\prime\prime}(x_0)=p_{n-1}^{\prime\prime}(x_n)=0$ \myfontsetting{8pt}{8pt}{\bf (自然スプライン補間)}}
        \begin{itemize}
            \item \myfontsetting{10pt}{10pt}{ 端条件を \myfontsetting{8pt}{8pt}{$p_0^{\prime}(x_0)=f^\prime(x_0)$, $p_{n-1}^{\prime}(x_n)=f^\prime(x_n)$}とすると{\bf 完全スプライン補間}}
        \end{itemize}
        %\item [\myfontsetting{12pt}{12pt}{\bf 方程式の数}] \myfontsetting{12pt}{12pt}{ $3n-2$ 本で変数の数に2だけ足りない}
    \end{itemize}
\end{itemize}
\end{frame}
%%%%%%%%%%%%%%%%%%%%%%%%%%%%%%%%
\begin{frame}
\frametitle{[問題] V\hspace{-.1em}I\hspace{-.1em}I-A}
%%%%% START_TAG A %%%%%
%\noindent{\bf [V\hspace{-.1em}I\hspace{-.1em}I. 関数近似と補間 (2)]}%RETURN

%\noindent{\bf V\hspace{-.1em}I\hspace{-.1em}I-A.}

\myfontsetting{18pt}{18pt}{
(1) 3つの制御点 $(x_0,y_0)=(0,1)$, $(x_1,y_1)=(1,2)$, $(x_2,y_2)=(4,2)$ を持つ1次のB-スプライン曲線を区分毎に $y=f(x)$ の形で求めよ。ただしノットベクトルは $\bm{t}=[0,0,1,2,2]$ を用いよ。\\
(2) 問1と同じ制御点を用いて2次のB-スプライン曲線を区分毎に $y=f(x)$ の形で求めよ。ただしノットベクトルは $\bm{t}=[0,0,0,1,1,1]$ を用いよ。
}
%%%%% END_TAG A %%%%%
\end{frame}
%%%%%%%%%%%%%%%%%%%%%%%%%%%%%%%%
\begin{frame}
\frametitle{[略解] V\hspace{-.1em}I\hspace{-.1em}I-A}
%\vspace{-0.5cm}
(1) \hspace{5mm} $y=\begin{cases}
x+1 & (x\in [0,1))\\
2 & (x\in [1,4])
\end{cases}$

\vspace{5mm}

(2) \hspace{5mm} $y=-\frac{1}{2} x + 3\sqrt{\frac{1}{2}\left(x + \frac{1}{2}\right)} - \frac{1}{2}$

\end{frame}
%%%%%%%%%%%%%%%%%%%%%%%%%%%%%%%%
\begin{frame}
\frametitle{{\large [手法] $k$ 次B-スプラインの構成 (1)}}
\begin{itemize}
    \setlength{\itemsep}{0.15cm}
    \item  \myfontsetting{15pt}{15pt}{ $n+1$ 個の制御点に対する $k$ 次のB-スプライン $B_{i,k}(t)$ \myfontsetting{10pt}{10pt}{ $(i=0,1,\dots,n)$} を構成する}
    \item \myfontsetting{15pt}{15pt}{ まず $n+k+2$ 次元のノットベクトル $\bm{t}=[t_0,t_1,\dots,t_{n+k+1}]$ を以下のように準備する}
    \begin{itemize}
        %\setlength{\itemsep}{0.15cm}
        \item \myfontsetting{15pt}{15pt}{
$\begin{cases}
t_i = 0 & \mathrm{if}\ i \le k\\
t_i = i-k &\mathrm{if}\ k < i \le n\\
t_i = n-k+1 & \mathrm{if}\  i > n
\end{cases}$
        \begin{itemize}
            \item [\myfontsetting{12pt}{12pt}{\bf 例}] \myfontsetting{15pt}{15pt}{ $n=2$, $k=1$ の時 $\bm{t}=[0,0,1,2,2]$}
        \end{itemize}
} 
    \end{itemize}
\end{itemize}
\end{frame}
%%%%%%%%%%%%%%%%%%%%%%%%%%%%%%%%
\begin{frame}
\frametitle{{\large [手法] $k$ 次B-スプラインの構成 (2)}}
\begin{itemize}
    \item \myfontsetting{15pt}{15pt}{ 0次のB-スプラインを以下のように定義}
    \begin{itemize}
        \item \myfontsetting{12pt}{12pt}{
$B_{i,0}(t) =
\begin{cases}
1 & \mathrm{if}\ t_i\le t < t_{i+1}\\
0 & \mathrm{otherwise}
\end{cases}$
}
    \end{itemize}
    \item  \myfontsetting{10pt}{10pt}{
$\displaystyle w_{i,k-1}(t) \equiv \frac{t -t_i}{t_{i+k}-t_i}$
} \myfontsetting{15pt}{15pt}{ とした時、漸化式
$B_{i,k}(t) = w_{i,k-1}(x) B_{i, k-1}(t) + \left\{1-w_{i+1,k-1}(x)\right\}B_{i+1,k-1}$ \\
を繰り返し用いて $B_{i,k}(t)$ を計算する}
    \item \myfontsetting{15pt}{15pt}{ $B_{i,k}(t)$ は以下の性質を満たす}
    \begin{itemize}
        \item \myfontsetting{12pt}{12pt}{ $\displaystyle \sum_{i=0}^n B_{i,k}(t) = 1$, $(t_0\le t < t_n)$}
        \item \myfontsetting{12pt}{12pt}{
$B_{i,k}(t) =
\begin{cases}
0 & \mathrm{if}\ t<t_i\ \mathrm{or}\ t \ge t_{i+k+1}\\
\mathrm{nonzero} & \mathrm{otherwise}
\end{cases}$
}
    \end{itemize}
\end{itemize}
\end{frame}
%%%%%%%%%%%%%%%%%%%%%%%%%%%%%%%%
\begin{frame}
\frametitle{{\large [手法] $k$ 次B-スプライン曲線}}

\begin{itemize}
    \item \myfontsetting{15pt}{15pt}{$k$ 次B-スプラインを $n+1$ 個の制御点 $p_i$ \myfontsetting{10pt}{10pt}{$(i=0,1,\dots,n)$} と掛け合わせて和を取ると $k$ 次B-スプライン曲線が得られる}
    \begin{itemize}
        \item \myfontsetting{15pt}{15pt}{
$\displaystyle\bm{r}(t) = \sum_{i=0}^{n} p_i B_{i,k}(t),\ t\in [t_0,t_n)$
}
    \end{itemize}
    \item \myfontsetting{15pt}{15pt}{ B-スプライン曲線自体は制御点を必ずしも通過しないため補間ではない}
    \item \myfontsetting{15pt}{15pt}{ 3次スプライン補間の基底として3次のB-スプラインを使用すると便利
    }
\end{itemize}
\end{frame}
%%%%%%%%%%%%%%%%%%%%%%%%%%%%%%%%
\begin{frame}
\frametitle{{\large [手法] $k$ 次B-スプライン構成の準備 (1)}}
    \begin{block}{{\bf\small ノットベクトル生成}
    \myfontsetting{13pt}{18pt}{ (knot vector generation)}}
        \myfontsetting{15pt}{18pt}{
        \begin{algorithmic}[1]
            \REQUIRE $n$, $k$
            \ENSURE $t_i$ \myfontsetting{10pt}{10pt}{ $(0\le i\le n)$}
            \FOR{$i=0,1,2,\dots,n+k+1$}
            \IF{$i \le k$}
            \STATE $t_i \leftarrow 0$
            \ELSIF{$k<i\le n$}
            \STATE $t_i \leftarrow i-k$
            \ELSE
            \STATE $t_i \leftarrow n-k+1$
            \ENDIF
            \ENDFOR
        \end{algorithmic}
        }
    \end{block}
\end{frame}
%%%%%%%%%%%%%%%%%%%%%%%%%%%%%%%%
\begin{frame}
\frametitle{{\large [手法] $k$ 次B-スプライン構成の準備 (2)}}
    \begin{block}{{\bf\small 0次B-スプライン}
    \myfontsetting{13pt}{18pt}{ ($k$-th order spline)}}
        \myfontsetting{15pt}{18pt}{
        \begin{algorithmic}[1]
            \REQUIRE $n$, $t$, $t_i$ \myfontsetting{10pt}{10pt}{ $(0\le i\le n)$}
            \ENSURE $B_{i,0}$ \myfontsetting{10pt}{10pt}{ $(0\le i\le n)$}
            \FOR{$i=0,1,2,\dots,n$}
            \IF{$t_i \le t < t_{i+1}$}
            \STATE $B_{i,0} \leftarrow 1$
            \ELSE
            \STATE $B_{i,0} \leftarrow 0$
            \ENDIF
            \ENDFOR
        \end{algorithmic}
        }
    \end{block}
\end{frame}
%%%%%%%%%%%%%%%%%%%%%%%%%%%%%%%%
\begin{frame}
\frametitle{{\large [手法] $k$ 次B-スプライン構成 (1)}}
    \begin{block}{{\bf\small $w_{i,k}$ \myfontsetting{10pt}{10pt}{ $(0\le i \le n)$} の計算}}
        \myfontsetting{15pt}{18pt}{
        \begin{algorithmic}[1]
            \REQUIRE $n$, $k$, $t$, $t_i$ \myfontsetting{10pt}{10pt}{ $(0\le i\le n)$}
            \ENSURE $w_{i,k}$ \myfontsetting{10pt}{10pt}{ $(0\le i\le n)$}
            \FOR{$i=0,1,2,\dots,n$}
            \IF{$t_{i+k+1} - t_i \neq 0$}
            \STATE $w_{i,k} \leftarrow \frac{t - t_i}{t_{i+k+1} - t_i}$
            \ELSE
            \STATE $w_{i,k} \leftarrow 0$
            \ENDIF
            \ENDFOR
        \end{algorithmic}
        }
    \end{block}
\end{frame}
%%%%%%%%%%%%%%%%%%%%%%%%%%%%%%%%
\begin{frame}
\frametitle{{\large [手法] $k$ 次B-スプライン構成 (2)}}
    \begin{block}{{\bf\small $B_{i,k}$ \myfontsetting{10pt}{10pt}{ $(0\le i\le n)$} の計算}}
        \myfontsetting{15pt}{18pt}{
        \begin{algorithmic}[1]
            \REQUIRE $n$, $k$, $w_{i,k}$, \myfontsetting{10pt}{10pt}{ $(0\le i\le n)$}
            \ENSURE $B_{i,k}$ \myfontsetting{10pt}{10pt}{ $(0\le i\le n)$}
            \FOR{$i=0,1,2,\dots,n-1$}
            \IF{$i<n$}
            \STATE $B_{i,k} \leftarrow w_{i, k-1}B_{i, k-1} + \left\{1 - w_{i+1, k-1}\right\} B_{i+1, k-1}$
            \ELSE
            \STATE $    B_{n,k}  \leftarrow  w_{n,k-1}B_{n,k-1}$
            \ENDIF
            \ENDFOR
        \end{algorithmic}
        }
    \end{block}
\end{frame}
%%%%%%%%%%%%%%%%%%%%%%%%%%%%%%%%
\begin{frame}
%%%%% PASTE_START_TAG B %%%%%
\frametitle{[問題] V\hspace{-.1em}I\hspace{-.1em}I-B}
%\noindent{\bf V\hspace{-.1em}I\hspace{-.1em}I-B.} 

\myfontsetting{12pt}{12pt}{
(1) 3次自然スプライン補間を用いて3つのデータ点 $(x_0,y_0)=(0,1)$, $(x_1,y_1)=(1,2)$, $(x_2,y_2)=(4,2)$ を通る2本の3次の区分的補間多項式 $p_i(x) = y_i + a_i (x-x_i) + b_i (x-x_i)^2 + c_i (x-x_i)^3$, $(i=0,1)$ を求めよ。%\\
}\\
\myfontsetting{12pt}{12pt}{
(2) 3次自然スプライン補間を用いて5つのデータ点 $(x_0,y_0)=(0,1)$, $(x_1,y_1)=(1,2)$, $(x_2,y_2)=(4,2)$, $(x_3,y_3)=(5,1)$, $(x_4,y_4)=(6,0)$ を通る4本の3次の区分的補間多項式 $p_i(x) = y_i + a_i (x-x_i) + b_i (x-x_i)^2 + c_i (x-x_i)^3$, $(i=0,1,2,3)$ を求めるプログラムを作成し、$a_i, b_i, c_i$ $(i=0,1,2,3)$ を有効数字10進3桁で4桁目を四捨五入して答えよ。
}
%%%%% PASTE_END_TAG B %%%%%
\end{frame}
%%%%%%%%%%%%%%%%%%%%%%%%%%%%%%%%
\end{document}
